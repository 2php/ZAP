\documentclass[11pt]{article}

%%%%%%%%%%%%%%%%%%%%%%%%%%%%%%%%%%%%%%%%%%%%%%%%%%%%%%%%%%%%%%%%%%%%%%%%%%%%%%%

\usepackage{graphicx}
\usepackage{textcomp}
\usepackage{parskip}
\usepackage{mathptmx}
\usepackage[margin=1.0in]{geometry}
\usepackage{array}
\usepackage{tabularx}
\usepackage{minted}
\usepackage{multirow}
\usepackage{textcomp}
\pagestyle{headings}

%%%%%%%%%%%%%%%%%%%%%%%%%%%%%%%%%%%%%%%%%%%%%%%%%%%%%%%%%%%%%%%%%%%%%%%%%%%%%%%

\begin{document}

\title{ ZAP : An ARM v4T Compatible Soft Processor }
\author{Revanth Kamaraj}

\begin{titlepage}
\clearpage\maketitle
\thispagestyle{empty}
\maketitle
\end{titlepage}

\begin{center}
\textbf{MIT License} \\
\end{center}
Copyright (c) 2016 Revanth Kamaraj (Email: revanth91kamaraj@gmail.com) \\\\
Permission is hereby granted, free of charge, to any person obtaining a copy
of this software and associated documentation files (the "Software"), to deal
in the Software without restriction, including without limitation the rights
to use, copy, modify, merge, publish, distribute, sublicense, and/or sell
copies of the Software, and to permit persons to whom the Software is
furnished to do so, subject to the following conditions: \\\\
The above copyright notice and this permission notice shall be included in all
copies or substantial portions of the Software. \\\\
THE SOFTWARE IS PROVIDED "AS IS", WITHOUT WARRANTY OF ANY KIND, EXPRESS OR
IMPLIED, INCLUDING BUT NOT LIMITED TO THE WARRANTIES OF MERCHANTABILITY,
FITNESS FOR A PARTICULAR PURPOSE AND NONINFRINGEMENT. IN NO EVENT SHALL THE
AUTHORS OR COPYRIGHT HOLDERS BE LIABLE FOR ANY CLAIM, DAMAGES OR OTHER
LIABILITY, WHETHER IN AN ACTION OF CONTRACT, TORT OR OTHERWISE, ARISING FROM,
OUT OF OR IN CONNECTION WITH THE SOFTWARE OR THE USE OR OTHER DEALINGS IN THE
SOFTWARE.

%%%%%%%%%%%%%%%%%%%%%%%%%%%%%%%%%%%%%%%%%%%%%%%%%%%%%%%%%%%%%%%%%%%%%%%%%%%%%%%
\pagebreak
\section{Introduction}

ZAP is an ARM{\textregistered} v4T compatible soft processor core. The code is
fully open source and is released under the MIT license.

%%%%%%%%%%%%%%%%%%%%%%%%%%%%%%%%%%%%%%%%%%%%%%%%%%%%%%%%%%%%%%%%%%%%%%%%%%%%%%%

\subsection{Features}

\begin{itemize}

\item \textbf{Fully ARM v4T compatible.}

        \begin{itemize}

        \item Executes the 32-bit wide ARM v4 instruction set.

        \item Executes the 16-bit wide compressed instruction set.

        \end{itemize}

\item \textbf{Deeper pipeline for better clock speed.}

        \begin{itemize}

        \item The processor is super-pipelined with a 9 stage pipeline to 
        achieve a high operating frequency. The pipeline has a data forwarding 
        capability to allow back to back instructions to execute without 
        stalls. Non trivial shifts require their operands a cycle early. Loads 
        have a 3 cycle latency and the pipeline will stall if an attempt is 
        made to access the register within the latency period.

        \end{itemize}

\item{\textbf{Supports interrupt and abort signaling}}

        \begin{itemize}

        \item Features dedicated high level sensitive IRQ, FIQ and memory abort 
        pins.

        \end{itemize}

\item \textbf{Can be interfaced with caches/MMU.}

        \begin{itemize}

        \item The CPSR of the processor is exposed as a port allowing for 
        implementation of a virtual memory system. 

        \item Memory stall may be indicates to the core via dedicated ports to 
        allow caches to be connected.

        \end{itemize}

\item \textbf{Coprocessor interface provided.}

        \begin{itemize}

        \item The coprocessor interface simply exposes internal signals of the 
        core. It is up to the coprocessor to interpret and process instructions 
        correctly.

        \end{itemize}

\item \textbf{Supports M-variant multiplication instructions}

        \begin{itemize}

        \item These instructions are supported: 

        \texttt
        {
                MUL, MLA, 
                SMULL, UMULL,
                SMLAL, UMLAL
        }

        \end{itemize}

\item \textbf{The core is configurable}

\begin{itemize}

        \item The processor may be synthesized without compressed instruction 
        support and/or coprocessor interface support to save area and improve
        speed.

\end{itemize}

\item \textbf{Designed for FPGA synthesis}

        \begin{itemize}

        \item Most memory structures of the processor map efficiently onto FPGA 
        block RAMs. The register file is overclocked by a 2x clock to allow for 
        2 write ports.

        \item The branch predictor memory also efficiently maps to FPGA block
        RAM.

        \item No device specific instantiations are made to allow for portability
        across FPGA vendors.

        \end{itemize}

\item \textbf{Faster performance of memory instructions}

        \begin{itemize}

        \item Memory instructions with writeback can be issued as a
        single instruction since the register file is built to have 2 write
        ports. This may improve performance.

        \end{itemize}

\item 
        \textbf{The processor core is written entirely in synthesizable 
        Verilog-2001.}

\item \textbf{A branch predictor is installed to compensate for the longer 
        pipeline.}

        \begin{itemize}

        \item Branches within a 2KB block of memory can be mapped into the 
        predictor without conflict. The predictor basically uses a bimodal
        prediction algorithm (2-bit saturating counter per branch entry).

        \end{itemize}

\item \textbf{Uses a base restored abort model}

        \begin{itemize}

        \item
        Uses a base restored abort model making it easier to write exception
        handlers. Basically, on a fault in between a multiple memory transfer, 
        the processor rolls back the base pointer register as it were before 
        the operation took place.

        \end{itemize}

\end{itemize}

%%%%%%%%%%%%%%%%%%%%%%%%%%%%%%%%%%%%%%%%%%%%%%%%%%%%%%%%%%%%%%%%%%%%%%%%%%%%%%%

\subsection{About This Manual}

The purpose of this manual is to document the processor core's design. This
document is very incomplete. I will try my best to update it.

%%%%%%%%%%%%%%%%%%%%%%%%%%%%%%%%%%%%%%%%%%%%%%%%%%%%%%%%%%%%%%%%%%%%%%%%%%%%%%%
\pagebreak
\section{Configuring the core and testbench}

Throughout, it is assumed that \texttt{\$ZAP\_HOME} points to the working
directory of the project.\\
Core/testbench configuration may be done using defines. The defines file is 
located in \\
\texttt{\$ZAP\_HOME/includes/config.vh}. \\

\begin{tabularx}{\textwidth}{| X | X | X | X |} 
 \hline
 \textbf{Define} & \textbf{Purpose} & \textbf{Required} for & \textbf{Comments} 
\\  
 \hline
         \multicolumn{4}{c}{\textbf{CORE CONFIGURATION}} \\
 \hline

 \texttt{THUMB\_EN} & Enabling compressed instruction support & Core Setup & 
 Enabling this increases core area and reduces performance. \\ 

 \texttt{COPROC\_IF\_EN} & Enabling coprocessor suport. Extra ports get added. &
  Core Setup & Enabling this increases core area and reduces performance. \\

 \hline
         \multicolumn{4}{c}{\textbf{TESTBENCH CONFIGURATION}} \\
 \hline

 \texttt{IRQ\_EN} & Generates periodic IRQ pulses. & Testbench & -- \\

 \texttt{SIM} & Generates extra messages. & Testbench & \emph{Must be UNDEFINED 
 for correct synthesis of the core in Xilinx since some debugging structures
in RTL are removed if this is not defined.} \\

 \texttt{VCD\_FILE\_PATH} & Set the path to the VCD data dump. & Testbench & -- \\

 \texttt{MEMORY\_IMAGE} & Path to the memory image Verilog file. & Testbench & 
 -- \\

 \texttt{MAX\_CLOCK\_CYCLES} & Set the number of cycles the simulation should 
 run before terminating. & Testbench & -- \\

 \texttt{SEED} & Set the testbench seed. The seed influences randomness. & 
 Testbench & -- \\

\hline
\end{tabularx}

%%%%%%%%%%%%%%%%%%%%%%%%%%%%%%%%%%%%%%%%%%%%%%%%%%%%%%%%%%%%%%%%%%%%%%%%%%%%%%%

\pagebreak
\section{Compiling Code and Running Simulation}

\subsubsection{Generating a binary using GNU tools}

You can use the existing GNU toolchain to generate code for the processor. This
section will briefly explain the procedure. For the purposes of this 
discussion, let us assume these are the source files...\\

\texttt{main.c} \\
\texttt{fact.c} \\
\texttt{startup.s} \\
\texttt{misc.s} \\
\texttt{linker.ld}  \emph{ This is the linker script.} \\\\

Generate a bunch of object files.\\

\texttt{arm-none-eabi-as -mcpu=arm7tdmi -g startup.s -o startup.o} \\
\texttt{arm-none-eabi-as -mcpu=arm7tdmi -g misc.s -o misc.o} \\
\texttt{arm-none-eabi-gcc -c -mcpu=arm7tdmi -g main.c -o main.o} \\
\texttt{arm-none-eabi-gcc -c -mcpu=arm7tdmi -g fact.c -o fact.o} \\\\

Link them up using a linker script...\\

\texttt{arm-none-eabi-ld -T linker.ld startup.o misc.o main.o fact.o -o 
prog.elf} \\\\

Finally generate a flat binary...\\
\texttt{arm-none-eabi-objcopy -O binary prog.elf prog.bin} \\\\

The .bin file generated is the flat binary.\\\\

%%%%%%%%%%%%%%%%%%%%%%%%%%%%%%%%%%%%%%%%%%%%%%%%%%%%%%%%%%%%%%%%%%%%%%%%%%%%%%%

\subsubsection{Generating a Verilog memory map}

\texttt{perl \$ZAP\_HOME/scripts/bin2mem.pl prog.bin prog.v}\\
The prog.v file looks like this...\\\\
\texttt{mem[0] = 8'b00;} \\
\texttt{mem[1] = 8'b01;} \\\\

%%%%%%%%%%%%%%%%%%%%%%%%%%%%%%%%%%%%%%%%%%%%%%%%%%%%%%%%%%%%%%%%%%%%%%%%%%%%%%%

\subsubsection{Invoking the simulator}
Ensure \texttt{config.vh} is set up correctly.

Your command must look like this...\\\\

\texttt{iverilog \$ZAP\_HOME/rtl/*.v \$ZAP\_HOME/*.v \$ZAP\_HOME/testbench/*.v
\$ZAP\_HOME/models/ram/ram.v -DSEED=22}\\\\

The rtl/*.v and rtl/*/*.v collect all of the synthesizable Verilog-2001 files,
the testbench/*.v collects all of the testbench (In this situations, the ram.v
file is a part of the testbench).

Provide some seed value (22 is used in the example). Ensure you edit the 
\texttt{config.vh} file before running the simulation to correctly point to the
memory map, vcd target output path etc for the simulator to pick up.

%%%%%%%%%%%%%%%%%%%%%%%%%%%%%%%%%%%%%%%%%%%%%%%%%%%%%%%%%%%%%%%%%%%%%%%%%%%%%%%

\subsection{Run Sample Code quickly...}

A sample .s and .c file is present in \texttt{\$ZAP\_HOME/sw/s} and 
\texttt{\$ZAP\_HOME/sw/c} respectively. To translate them to binary and to a 
Verilog memory map, you can run the Perl script (See NOTE below) \\\\ 

\texttt{perl \$ZAP\_HOME/debug/run\_sim.pl} \\\\ 

\textbf{NOTE:} Ensure you set all the variables in the Perl script as per the 
table below...\\\\

% This table tells what to place in script variables.
\begin{tabularx}{\textwidth}{| X | X |}
 \hline
        \textbf{Variable} & \textbf{Purpose} \\
 \hline

 \texttt{ZAP\_HOME} & Set this to the project working directory.\\\\

 \texttt{LOG\_FILE\_PATH} & Set this to the place where you want the log file to 
 be created.\\\\

 \texttt{ASM\_PATH} & Set this to the location of your startup assembly file.
 \\\\

 \texttt{C\_PATH} & Set this to the location of your C file. \\\\

 \texttt{LINKER\_PATH} & Set this to the location of the linker script. \\\\

 \texttt{TARGET\_BIN\_PATH} & Set this to the target bin file location. \\\\

 \texttt{VCD\_PATH} & Set this to location where the VCD is to be created.
 \textbf{This must match what is in config.vh} \\\\

 \texttt{MEMORY\_IMAGE} & Set this to the location where the memory image must be
 created. 
 \textbf{This must match what is in config.vh} \\\\

 \hline
\end{tabularx}

%%%%%%%%%%%%%%%%%%%%%%%%%%%%%%%%%%%%%%%%%%%%%%%%%%%%%%%%%%%%%%%%%%%%%%%%%%%%%%%

\pagebreak
\section{Top Level Description}

\end{document}

